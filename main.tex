\documentclass[12pt]{article}

\usepackage{sbc-template}
\usepackage{graphicx,url}
\usepackage[utf8]{inputenc}
\usepackage[brazil]{babel}
\usepackage{booktabs}
\usepackage{tabularx}
\usepackage[justification=centering]{caption} 
\usepackage{amsmath}
\usepackage{float}
\usepackage{algorithm}	
\usepackage{algpseudocode}
\usepackage[T1]{fontenc}
\usepackage{multirow} 
\usepackage{threeparttable}
   
\sloppy

\title{Implementação de uma Árvore de Decisão Baseada em Algoritmo Genético e Implementação de uma Árvore de Decisão para Aprimoramento de Performance do KNN para a Detecção de Apneia Obstrutiva do Sono via Variabilidade de Frequência Cardíaca}

\author{Thiago de Matos \inst{1}, Gabriel Branco\inst{2}, José Baranauskas\inst{1}}
\address{
Universidade de São Paulo (USP) \\
Faculdade de Filosofia, Ciências e Letras de Ribeirão Preto (FFCLRP)\\ Departamento de Computação e Matemática (DCM)
 \nextinstitute
 Universidade de São Paulo (USP) \\
Faculdade de Filosofia, Ciências e Letras de Ribeirão Preto (FFCLRP)\\ Física Aplicada à Medicina e Biologia (FAMB)
\email{v.thiago@usp.br, gabriel.branco@usp.br, augusto@usp.br}
}

\begin{document} 

\maketitle

\begin{resumo} 
Neste trabalho, abordamos o problema da detecção da Apneia Obstrutiva do Sono (AOS) em diferentes pacientes por meio de técnicas de aprendizado de máquina, utilizando dados de variabilidade da frequência cardíaca, saturação de oxigênio no sangue e características antropométricas do paciente. Para isso, dois algoritmos foram desenvolvidos. O primeiro consiste em uma Árvore de Decisão (AD) cuja construção é guiada por um Algoritmo Genético (AG). Nesse primeiro modelo, o AG é responsável por selecionar a melhor premissa para a divisão de cada nó da árvore, com o objetivo de minimizar o erro majoritário. Essa abordagem substitui a tradicional busca exaustiva, que avalia todas as possíveis divisões com base na média entre valores consecutivos dos atributos, possibilitando explorar diferentes maneiras de particionar os dados. O segundo método, denominado DTKNN (Decision Tree and K-Nearest Neighbors), integra o KNN tradicional com uma etapa de seleção adaptativa de atributos por meio de uma AD. Em cada \textit{fold} de um processo de validação cruzada, uma AD é treinada no subconjunto de treino para selecionar, segundo um \textit{threshold}, os atributos mais frequentes, que compõem o vetor de entrada do KNN. Assim, o DTKNN busca eliminar atributos redundantes ou pouco informativos de forma dinâmica, ajustando-se a cada partição dos dados. Ambos os algoritmos tiveram suas performances comparadas às versões tradicionais na tarefa de classificação da AOS.
\end{resumo}

\input{Sections/Introducao}

\input{Sections/Trabalhos_Relacionados}

\input{Sections/Metodologia}

\input{Sections/Resultados}

\input{Sections/Conclusão}

\bibliographystyle{sbc}
\bibliography{referencias}

\appendix
\clearpage  

\input{Sections/Apêndice}
\end{document}
